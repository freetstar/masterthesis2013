% !Mode:: "TeX:UTF-8"


\ctitle{基于MPI的随机算法的并行实现}  %封面用论文标题,自己可手动断行
\etitle{Title: Parallel computing of Some Randomized algorithms using MPI}
\caffil{天津师范大学计算机与信息工程学院} %学院名称
\csubjecttitle{学科专业}
\csubject{计算机技术}   %专业
\cauthortitle{研究生}     % 学位
\cauthor{连高欣}   %学生姓名
\csupervisortitle{指导教师}
\csupervisor{张少强~~教授} %导师姓名

\declaretitle{独创性声明}
\declarecontent{
本人声明所呈交的学位论文是本人在导师指导下进行的研究工作和取得的研究成果,除了文中特别加以标注和致谢之处外,论文中不包含其他人已经发表或撰写过的研究成果,也不包含为获得 {\underline{\kai\textbf{~天津师范大学~}}}或其他教育机构的学位或证书而使用过的材料。与我一同工作的同志对本研究所做的任何贡献均已在论文中作了明确的说明并表示了谢意。
}
\authorizationtitle{学位论文版权使用授权书}
\authorizationcontent{
本学位论文作者完全了解{\underline{\kai\textbf{~天津师范大学~}}}有关保留、使用学位论文的规定。特授权{\underline{\kai\textbf{~天津师范大学~}}} 可以将学位论文的全部或部分内容编入有关数据库进行检索,并采用影印、缩印或扫描等复制手段保存、汇编以供查阅和借阅。同意学校向国家有关部门或机构送交论文的复印件和磁盘。
}
\authorizationadd{(保密的学位论文在解密后适用本授权说明)}
\authorsigncap{学位论文作者签名:}
\supervisorsigncap{导师签名:}
\signdatecap{签字日期:}


\cdate{\CJKdigits{\the\year} 年\CJKnumber{\the\month} 月 \CJKnumber{\the\day} 日}
% 如需改成二零一二年四月二十五日的格式,可以直接输入,即如下所示
% \cdate{二零一二年四月二十五日}
% \cdate{\the\year 年\the\month 月 \the\day 日} % 此日期显示格式为阿拉伯数字 如2012年4月25日
\cabstract{
随着计算机技术的不断发展,各类大数据量的计算要求不断浮现,越来越多的学科领域,如生物,气象,
医疗,基因测序比对等领域开始使用计算机和数值模拟的方法来解决科学和工程实践中遇到的各种问题,社会需求对
计算机的性能提出了重要的要求. 这些需求具有大计算量,时间周期要求短,交换量大等特点.尽管单台计算机能够
满足日常使用,具有不错的性能和可靠性,但是例如云计算,大量卫星云图计算的大计算量对计算机性能提出了
为重要的要求.通过并行计算的技术,可以实现多台单机构成计算群的方式来提供大量计算能力,用多台计算机模拟和
实现计算性能更为强大的计算机,满足高性能计算需求.

    本课题着重研究了MPI在计算机并行计算技术中的应用,分析和实现了MPI并行计算模型.MPI实现
消息传递编程的标准,具有实现灵活,可移植性强,性能高等特点.MPI以语言独立的方式定义了库
接口,提供了c/c++等多种语言的绑定,是目前最高效率,非常流行的并行计算平台.

    本文首先对并行计算进行讨论,阐述了并行计算体系结构,常用并行计算机,并行编程理论和并行算法.然后
介绍了消息传递编程标准的MPI,构建了MPI运行环境,研究和实现了多个随机性常用算法,通过实现将多个单机程序
进行并行化的方法,介绍了MPI的详细应用,在Linux服务器运行了MPI程序,统计了程序运行时间,得出MPI运行效率,
证明了并行计算的高效,实现了单机程序到并行程序的转换.
}

\ckeywords{[并行计算]  [MPI]  [常用算法] [并行化] }

\eabstract{

}

\ekeywords{"parallel computing" , "MPI" , "alogrithm" , "parallelize" }

\makecover

\clearpage 
